% slides.tex — Lund University inspired title slide + content slide mock
\documentclass[aspectratio=169]{beamer}

% keep presentation lightweight and self-contained
\usetheme{default}
\setbeamertemplate{navigation symbols}{}
\setbeamertemplate{headline}{}

\usepackage{tikz}
\usetikzlibrary{calc}
\usepackage{graphicx}
\usepackage{ragged2e}
\usepackage{calc}
\usepackage{xcolor}
\usepackage{iftex}

\ifPDFTeX
  \usepackage[T1]{fontenc}
  \usepackage[utf8]{inputenc}
  \usepackage{mathptmx} % Times-like serif
  \usepackage[scaled=0.92]{helvet} % Arial-like sans
  \newcommand{\TitleFont}{\rmfamily}
  \newcommand{\SubtitleFont}{\sffamily}
\else
  \usepackage{fontspec}
  \setmainfont{Times New Roman}
  \newfontfamily\TitleFont{Times New Roman}[Scale=1.0]
  \newfontfamily\SubtitleFont{Arial}[Scale=1.0]
\fi

% --- Color palette sampled from Presentation-template-page1.jpg ---
\definecolor{LUBeige}{HTML}{D8CCAA}
\definecolor{LUBronze}{HTML}{8A4F16}

% ensure other frames revert to a neutral background
\setbeamercolor{background canvas}{bg=white}
\setbeamercolor{normal text}{fg=black}
\setbeamercolor{footline}{fg=LUBronze,bg=white!0}


\newdimen\CircleRadius
\CircleRadius=0.46\paperheight
\newlength{\ContentHeight} % matched to left-column content height on content slides
\newlength{\ImageHeight}
\newlength{\ImageWidth}
\newlength{\ImageInnerHeight}
\newlength{\ImageInnerWidth}

\title{Module 2: Introduction to low-code Python programming}
%\subtitle{Lesson 2.2: Python and AI-aided data analysis \\[0.8em]AI-aided content analysis of sustainability communication \\[0.8em]nils.holmberg@iko.lu.se}
\subtitle{Lesson 2.2: Python and AI-aided data analysis \\[0.8em]nils.holmberg@iko.lu.se}
\author{}

% Custom footline with frame number (left) and compact logo (right)
\setbeamertemplate{footline}{%
  \vspace{0.35cm}% raises the footer slightly above the slide edge
  \begin{beamercolorbox}[wd=\paperwidth,ht=3ex,dp=1.2ex,leftskip=0.9cm,rightskip=0.9cm]{footline}
    {\SubtitleFont\small\color{LUBronze}\insertframenumber}
    \hfill
    \raisebox{-0.1cm}{\includegraphics[height=0.75cm]{edulab-lulogo.png}}
  \end{beamercolorbox}%
}

% Bronze round bullets with matching text color for itemize blocks
\setbeamercolor{itemize item}{fg=LUBronze}
\setbeamercolor{itemize body}{fg=LUBronze}
\setbeamertemplate{itemize item}{\color{LUBronze}\scriptsize\raise0.2ex\hbox{\textbullet}}

\makeatletter
\defbeamertemplate*{title page}{lu}{%
  \begin{tikzpicture}[remember picture,overlay]
    % full-bleed beige background — adjust offsets here if you need more/less margin.
    \fill[LUBeige] (current page.south west) rectangle (current page.north east);

    % Lund University logo in the upper-left — tweak the shifts to nudge placement.
    \node[anchor=north west] at ([xshift=0.3cm,yshift=-0.25cm]current page.north west) {
      \includegraphics[height=1.9cm]{edulab-lulogo.png}
    };

    % Circular pelican artwork, bottom cropped by slide edge — adjust center/radius as needed.
    \coordinate (circlecenter) at ([xshift=0.36\paperwidth,yshift=0.22\paperheight]current page.south west);
    \begin{scope}
      \clip (circlecenter) circle (\CircleRadius);
      \node at (circlecenter) {
        \includegraphics[width=1.12\paperheight]{pelican.png}
      };
    \end{scope}

    % Title block positioned on the right — tweak node shift/inner sep to fine-tune spacing.
    \node[anchor=north east] at ([xshift=-1.0cm,yshift=-1.0cm]current page.north east) {
      \begin{tikzpicture}
        \node[anchor=north east, fill=white, inner sep=18pt, minimum width=0.48\paperwidth] (titlebox) at (0,0) {
          \begin{minipage}{0.44\paperwidth}
            \raggedright
            {\TitleFont\fontsize{14}{16}\selectfont\bfseries\color{LUBronze}\inserttitle\par}
            \vspace{0.5em}
            {\color{LUBronze}\rule{\linewidth}{0.8pt}\par}
            \vspace{0.4em}
            {\SubtitleFont\fontsize{11}{13}\selectfont\color{LUBronze}\insertsubtitle\par}
          \end{minipage}
        };
      \end{tikzpicture}
    };
  \end{tikzpicture}
}
\setbeamertemplate{title page}[lu]
\makeatother

\begin{document}

\setbeamercolor{background canvas}{bg=LUBeige}
\begin{frame}[plain]
  \titlepage
\end{frame}
\setbeamercolor{background canvas}{bg=white}
\setcounter{framenumber}{0} % so the first content slide starts at 1

\begin{frame}[t]{}
  \vspace*{0.5cm}
  {\TitleFont\fontsize{18}{22}\selectfont\color{LUBronze}Python for Social Science Data Analysis\par}
  \vspace{0.3em}
  {\color{LUBronze}\rule{\linewidth}{0.8pt}}\par
  \vspace{0.2cm}
  \begin{columns}[t]
    \begin{column}[t]{0.4\textwidth}
      \vspace*{0pt}
      \begin{itemize}\setlength\itemsep{0.65em}
        \item Descriptive inferential exploratory overview
        \item Highlight quantitative social methods
        \item Introduce text image analysis
        \item Python enables flexible scaling
        \item Use pandas matplotlib scikit
      \end{itemize}
    \end{column}
    \begin{column}[t]{0.4\textwidth}
      \vspace*{0pt}
      \includegraphics[
        width=\linewidth,
        height=0.60\textheight,
        keepaspectratio=false,
        trim=0 0 0 0,
        clip
      ]{2-2-1-video-lecture-slides-1.png}
    \end{column}
  \end{columns}
\end{frame}

\begin{frame}[t]{}
  \vspace*{0.5cm}
  {\TitleFont\fontsize{18}{22}\selectfont\color{LUBronze}AI-Aided Quantitative Analysis with Python\par}
  \vspace{0.3em}
  {\color{LUBronze}\rule{\linewidth}{0.8pt}}\par
  \vspace{0.2cm}
  \begin{columns}[t]
    \begin{column}[t]{0.4\textwidth}
      \vspace*{0pt}
      \begin{itemize}\setlength\itemsep{0.65em}
        \item LLMs generate statistical code
        \item Query data with prompts
        \item Automate tests and models
        \item Lower barriers for newcomers
        \item Balance manual control AI
      \end{itemize}
    \end{column}
    \begin{column}[t]{0.4\textwidth}
      \vspace*{0pt}
      \includegraphics[
        width=\linewidth,
        height=0.60\textheight,
        keepaspectratio=false,
        trim=0 0 0 0,
        clip
      ]{2-2-1-video-lecture-slides-1.png}
    \end{column}
  \end{columns}
\end{frame}

\begin{frame}[t]{}
  \vspace*{0.5cm}
  {\TitleFont\fontsize{18}{22}\selectfont\color{LUBronze}AI-Aided Computational Content Analysis\par}
  \vspace{0.3em}
  {\color{LUBronze}\rule{\linewidth}{0.8pt}}\par
  \vspace{0.2cm}
  \begin{columns}[t]
    \begin{column}[t]{0.4\textwidth}
      \vspace*{0pt}
      \begin{itemize}\setlength\itemsep{0.65em}
        \item LLMs classify summarize topics
        \item Image models study visuals
        \item Merge text visual features
        \item Automate entity theme detection
        \item Scale massive corpus analysis
      \end{itemize}
    \end{column}
    \begin{column}[t]{0.4\textwidth}
      \vspace*{0pt}
      \includegraphics[
        width=\linewidth,
        height=0.60\textheight,
        keepaspectratio=false,
        trim=0 0 0 0,
        clip
      ]{2-2-1-video-lecture-slides-1.png}
    \end{column}
  \end{columns}
\end{frame}

\begin{frame}[t]{}
  \vspace*{0.5cm}
  {\TitleFont\fontsize{18}{22}\selectfont\color{LUBronze}Low-Code Data Analysis with Tables\par}
  \vspace{0.3em}
  {\color{LUBronze}\rule{\linewidth}{0.8pt}}\par
  \vspace{0.2cm}
  \begin{columns}[t]
    \begin{column}[t]{0.4\textwidth}
      \vspace*{0pt}
      \begin{itemize}\setlength\itemsep{0.65em}
        \item View data with pandas
        \item Filter group summarize quickly
        \item Compare high-code low-code
        \item AI generates interprets tables
        \item Use pandasgui datatable ChatGPT
%        \item Lab: download Colab notebook
      \end{itemize}
    \end{column}
    \begin{column}[t]{0.4\textwidth}
      \vspace*{0pt}
      \includegraphics[
        width=\linewidth,
        height=0.60\textheight,
        keepaspectratio=false,
        trim=0 0 0 0,
        clip
      ]{2-2-1-video-lecture-slides-4.png}
    \end{column}
  \end{columns}
\end{frame}

\begin{frame}[t]{}
  \vspace*{0.5cm}
  {\TitleFont\fontsize{18}{22}\selectfont\color{LUBronze}Data Transformations with AI Assistance\par}
  \vspace{0.3em}
  {\color{LUBronze}\rule{\linewidth}{0.8pt}}\par
  \vspace{0.2cm}
  \begin{columns}[t]
    \begin{column}[t]{0.4\textwidth}
      \vspace*{0pt}
      \begin{itemize}\setlength\itemsep{0.65em}
        \item Select variables via prompts
        \item Filter cases by conditions
        \item Aggregate summaries by groups
        \item Contrast pandas and AI queries
        \item Learn with interactive AI
      \end{itemize}
    \end{column}
    \begin{column}[t]{0.4\textwidth}
      \vspace*{0pt}
      \includegraphics[
        width=\linewidth,
        height=0.60\textheight,
        keepaspectratio=false,
        trim=0 0 0 0,
        clip
      ]{2-2-1-video-lecture-slides-4.png}
    \end{column}
  \end{columns}
\end{frame}

\begin{frame}[t]{}
  \vspace*{0.5cm}
  {\TitleFont\fontsize{18}{22}\selectfont\color{LUBronze}Low-Code Data Analysis with Graphs\par}
  \vspace{0.3em}
  {\color{LUBronze}\rule{\linewidth}{0.8pt}}\par
  \vspace{0.2cm}
  \begin{columns}[t]
    \begin{column}[t]{0.4\textwidth}
      \vspace*{0pt}
      \begin{itemize}\setlength\itemsep{0.65em}
        \item Create basic bar line hist
        \item Contrast univariate bivariate plots
        \item AI suggests suitable charts
        \item Use matplotlib seaborn plotnine
        \item Reveal trends with little setup
%        \item Lab: convert notebook Quarto
      \end{itemize}
    \end{column}
    \begin{column}[t]{0.4\textwidth}
      \vspace*{0pt}
      \includegraphics[
        width=\linewidth,
        height=0.60\textheight,
        keepaspectratio=false,
        trim=0 0 0 0,
        clip
      ]{2-2-1-video-lecture-slides-6.png}
    \end{column}
  \end{columns}
\end{frame}

\begin{frame}[t]{}
  \vspace*{0.5cm}
  {\TitleFont\fontsize{18}{22}\selectfont\color{LUBronze}Multivariate and Interactive Graphs\par}
  \vspace{0.3em}
  {\color{LUBronze}\rule{\linewidth}{0.8pt}}\par
  \vspace{0.2cm}
  \begin{columns}[t]
    \begin{column}[t]{0.4\textwidth}
      \vspace*{0pt}
      \begin{itemize}\setlength\itemsep{0.65em}
        \item Show multivariable relationships
        \item Use scatter heatmap bubble
        \item Add interactivity with plotly altair
        \item AI refines layouts variables
        \item Explore patterns dynamically
      \end{itemize}
    \end{column}
    \begin{column}[t]{0.4\textwidth}
      \vspace*{0pt}
      \includegraphics[
        width=\linewidth,
        height=0.60\textheight,
        keepaspectratio=false,
        trim=0 0 0 0,
        clip
      ]{2-2-1-video-lecture-slides-6.png}
    \end{column}
  \end{columns}
\end{frame}

\end{document}